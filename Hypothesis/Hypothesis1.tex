\documentclass[a4paper,12pt]{article}
\usepackage[hidelinks]{hyperref}
\usepackage[margin=1in]{geometry}
\usepackage{amsmath}
\usepackage{amssymb}
\usepackage{graphicx}
\usepackage{hyperref}
\usepackage{enumitem}
\usepackage{natbib}
\bibliographystyle{../../readings/acl_natbib}

\begin{document}

\title{Hypothesis - Memorization in the ACT Model}
\author{}
\date{}
\maketitle

\section*{Information Flow in ACT Model}
\textbf{Input:} Images from two cameras: $I_{\text{top}}, I_{\text{lat}} \in \mathbb{R}^{480 \times 640 \times 3}$ and joint angles $q \in \mathbb{R}^6$. Images are processed by ResNet: 
\[ V_{\text{top}} = \text{ResNet}(I_{\text{top}}), \quad V_{\text{lat}} = \text{ResNet}(I_{\text{lat}}), \quad V_{\text{top}}, V_{\text{lat}} \in \mathbb{R}^{300 \times 512}.
\]

\textbf{Tokenization:} Flattened camera features ($600$ "pixel tokens" - 300 for each camera), arm state token\footnote{Why not create six tokens, one for each joint angle? This might help the model better understand the arm's configuration.} 
($q \in \mathbb{R}^{512}$), and condition token
\footnote{We focuses here only on the model's inference process. During training, an additional encoder processes samples from the teleoperation setup, producing a latent variable $z$. This variable serves as a condition for the decoder, which takes the inputs described here. The model's objective is to reconstruct teleoperation actions — a non-trivial choice, as the CVAE framework was used instead of a more direct approach to predict the next steps.}($c \in \mathbb{R}^{512}$) form $602$ tokens of dimension $512$.

\textbf{Transformer Encoding:} The $602$ tokens are processed through $3$ Transformer encoder layers, enabling global information flow:
\[ 602 \times 512 \rightarrow 602 \times 512.
\]

\textbf{Action Vector Update:} A zero-initialized action vector ($a \in \mathbb{R}^{512}$) is updated using cross-attention with encoder output and an MLP:
\[ a \in \mathbb{R}^{512} \rightarrow \text{Cross-attention with the 602 tokens from the encoder} \rightarrow \text{MLP} \rightarrow \mathbb{R}^{6} \text{ (robot movement)}.
\]

\section*{Hypothesis: Memorization in ACT}
\textbf{Hypothesis:} The MLP layer in the decoder part of the ACT (\cite{Zhao2023Apr}) model memorize exact actions instead of generalizing. 
During inference, \textbf{the action vector after cross-attention \footnote{
Observation: Attention maps suggest only a subset of tokens (5-20) receive high attention scores, varying across different input images. This indicates that during encoding, environmental information is represented by a small subset of the $602$ tokens. Interestingly, the arm state token is not among those receiving high weights, implying that pixel tokens absorb and convey necessary arm state information to the action vector.
} contains complete environmental information (e.g., cube location, gripper state) even of unseen training examples}. 
However, the MLP acts as a "lookup table" of actions rather than applying general rules for novel scenarios.
(This is based on observationa from \cite{Geva2020Dec,Meng2022Feb} that the mlp layers in transformers is where memories are stored.)

\section*{Validation Plan}
1.\textbf{Probing Action Vector:} Train linear classifiers on the action vector to verify if it encodes all relevant environmental details.

2. \textbf{Generalization to Novel Scenes:} Test whether the action vector generalizes to new, unseen cube positions.

3. \textbf{Visualization:} Adapt techniques from \cite{Toker2024Mar} ("Diffusion lens) to visualize hidden states of the action vector and analyze stored information.

\textbf{Future Steps:} Investigate whether the MLP memorizes steps in a specific environment representation or if it applies generalizable rules.

\bibliography{../../readings/citations}

\end{document}
